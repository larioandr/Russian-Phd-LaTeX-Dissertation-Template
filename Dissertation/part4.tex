\chapter{Анализ производительности опорной беспроводной сети}\label{ch:ch4}


\section{Сеть массового обслуживания с линейной топологией}\label{sec:ch4_queueing_network}

\subsection{Элементарная модель: линейная сеть с узлами M/M/1}\label{sec:ch4_mm1_network}

\subsection{Линейная сеть с узлами MAP/PH/1/M}

\subsection{Приближенные методы расчёта характеристик сети с узлами MAP/PH/1/M}



\section{Моделирование задержки в канале}\label{sec:ch4_service_time}

\subsection{Обзор методов моделирования времени передачи пакетов в беспроводном канале}

\subsection{Методика оценки времени передачи пакетов}



\section{Численный расчёт характеристик многошаговой беспроводной сети}\label{sec:ch4_numeric_results}

\subsection{Построение PH-распределений, моделирующих длительность задержек в беспроводных каналах}



\section{Экспериментальная оценка пропускной способности сети}\label{sec:ch4_stand_results}



\section{Заключение}\label{sec:ch3_conclusion}



\clearpage
